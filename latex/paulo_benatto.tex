% LaTeX file for resume 
% This file uses the resume document class (res.cls)

\documentclass[margin]{res}
\usepackage [brazil]{babel}     % nomes e hifenaçã em português
 
\usepackage{t1enc}              % Permite digitar os acentos de forma normal
\usepackage[utf8]{inputenc} 

\topmargin=-0.5in  % start text higher on the page
\setlength{\textheight}{10in} % increase text height to fit resume on 1 page
\begin{document}  
\name{\textit{Paulo Leonardo Benatto}}

\address{Brighton, UK \\ benatto@gmail.com \\ Phone: +44 07424600850 \\ Post Code: BN21HF }
                           
                        
\begin{resume}
 
\section{Summary}      I am Paulo Leonardo Benatto, a Brazilian coder working as Linux 
System Administrator at Brandwatch. Most of my profissional life I spent developing software
in C language over Linux platform. In this moment I'm looking for new challenges and I have 
big interest to work with Golang, Python, C and Lua. 
 
\section{Education}	Universidade Estadual do Oeste do Parana, BSc in Computer Science, December 2007.
  
\section{Experience}

\vspace{-0.1in}
   \begin{tabbing}
   \hspace{2.3in}\= \hspace{1.7in}\= \kill % set up two tab positions
    \textbf{Brandwatch}    \>\>\textbf{Jul 2014 - Present}\\
    \textit{Linux Systems Administrator}\\        
    \textbf{Main Technologies}: Linux, Python, Puppet, Bareos, Ansible, Git;
   \end{tabbing}\vspace{-20pt}      % suppress blank line after tabbing
    \vspace{2mm}
I try to keep all Debian GNU/Linux servers, physical and virtual, running smoothly as possible.
I enjoy automating tasks using ansible. Sometimes I have to code in Python to make my life easier.
In my spare time I like to study about programming languages such as Python, Golang and C.

\vspace{-0.1in}

\vspace{-0.1in}
   \begin{tabbing}
   \hspace{2.3in}\= \hspace{1.7in}\= \kill % set up two tab positions
    \textbf{DBA}    \>\>\textbf{Dec 2013 - Feb 2014}\\
    \textit{Software Engineer}\\        
    \textbf{Main Technologies}: Linux, Python, Raspberry PI and C (GCC, Valgrind, Splint);
   \end{tabbing}\vspace{-20pt}      % suppress blank line after tabbing
    \vspace{2mm}
        I was a freelance Software Engineer for 3 months and my main dutie was develop
        a shared library using C language and Linux envirtonment. The project intended to 
        analyse vehicle traffic on Brazilian highways.
\vspace{-0.1in}

   \begin{tabbing}
   \hspace{2.3in}\= \hspace{1.7in}\= \kill % set up two tab positions
    \textbf{SEC+}    \>\>\textbf{Dec 2012 - Dec 2013}\\
    \textit{Software Engineer}\\        
    \textbf{Main Technologies}: Linux, Python, Django Framework, JavaScript and C;
   \end{tabbing}\vspace{-20pt}      % suppress blank line after tabbing
    \vspace{2mm}
     Backend development of web system for intelligent monitoring and management of natural disasters
     using Python and the Django framework. Front-end with Javascript (JQuery, Bootstrap, Google Maps API),
     JSON, HTML5, CSS.

   \begin{tabbing}
   \hspace{2.3in}\= \hspace{1.7in}\= \kill % set up two tab positions
    \textbf{Digitro Technology - NDS}    \>\>\textbf{Jan 2012 - Dec 2012}\\
    \textit{Software Engineer}\\   
    \textbf{Main Technologies}: Linux, C/C++, Wireshark, ShellScript;
   \end{tabbing}\vspace{-20pt}      % suppress blank line after tabbing
    \vspace{2mm}
    Member of the team responsible to design and develop the roadmap features of a product called
    Guardião used to interecept calls and internet trafic of people whose are being investigated by the police. 
    Specifically allocated on the design and development of a module which main goal is to intercept the actions 
    done on Facebook, WhatsApp and Hotmail of a target, using the gathered data to cross reference with another
    possible targets.
   
   %\vspace{1mm}
   \begin{tabbing}
   \hspace{2.3in}\= \hspace{1.5in}\= \kill % set up two tab positions
    \textbf{Digitro Technology - STE}    \>\>\textbf{Set 2008 - Dec 2011}\\
    \textit{Software Engineer}\\   
    \textbf{Main Technologies}: Linux, VoIP, C/C++, shellscript, protocols: UDP, TCP, SIP;
   \end{tabbing}\vspace{-20pt}      % suppress blank line after tabbing
    \vspace{2mm}
    
    Member of a team responsible to design and develop VoIP products such as PBX, softphone and IP phone. 
    All projects was developed using C language in Linux and Windows environment.
    

\section{Skills Base}  \textit{Operating System}:  Linux (Debian, Ubuntu, CentOS), Windows NT/XP/Vista/7 and OSX;

			\textit{Networkings}: TCP/IP protocol suite;
  
			\textit{Progamming Languages}: C, Go, Pascal, Python, JavaScript, plus some experience with Lua and Java;
  
			\textit{Virtualization}: VirtualBox, VMWare, plus some experience with Xen;

			\textit{Languages}: Fluent in Portuguese, Intermediate in English and Spanish;
 
\section{Open Source Projects}
		\begin{itemize}
		    \vspace{2mm}
		    \item \textbf{libpenetra}: The libpenetra was created with the goal of studying the windows binary format 
		                               known as Portable Executable (PE). With libpenetra you can access all information
		                               about PE binaries. (\texttt{https://github.com/patito/libpenetra}) \vspace{1mm}
		                               
		    \item \textbf{libmalelf}: The libmalelf is an evil library that SHOULD be used for good! It was developed
		                              with the intent to assist in the process of infecting binaries and provide a safe 
		                              way to analyze malwares. (\texttt{https://github.com/SecPlus/libmalelf})\vspace{1mm}
		                              
		    \item \textbf{malelf}: Malelf is a tool that uses libmalelf to dissect and infect ELF binaries. 
		                           (\texttt{https://github.com/SecPlus/malelf})
		\end{itemize}
 
\section{More Info}
    \begin{itemize}
        \item \textbf{Linkedin}: http://www.linkedin.com/in/benatto
        \item \textbf{Github}: https://github.com/patito
        \item \textbf{Blog}: patito.github.io
    \end{itemize}


\end{resume} 
\end{document}









